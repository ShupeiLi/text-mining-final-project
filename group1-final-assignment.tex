% This is samplepaper.tex, a sample chapter demonstrating the
% LLNCS macro package for Springer Computer Science proceedings;
% Version 2.21 of 2022/01/12
%
\documentclass[runningheads]{llncs}
%
\usepackage[T1]{fontenc}
% T1 fonts will be used to generate the final print and online PDFs,
% so please use T1 fonts in your manuscript whenever possible.
% Other font encondings may result in incorrect characters.
%
\usepackage{graphicx}
% Used for displaying a sample figure. If possible, figure files should
% be included in EPS format.
%
% If you use the hyperref package, please uncomment the following two lines
% to display URLs in blue roman font according to Springer's eBook style:
%\usepackage{color}
%\renewcommand\UrlFont{\color{blue}\rmfamily}
%
\begin{document}
%
\title{A BERT-Based Ensemble Learning Approach for Sentiment Classification in Twitter\thanks{Text Mining, Master CS, Fall 2022, Leiden, the Netherlands}}
%
%\titlerunning{Abbreviated paper title}
% If the paper title is too long for the running head, you can set
% an abbreviated paper title here
%
\author{Shupei Li\inst{1}\and
Ziyi Xu\inst{1}}
%
%\authorrunning{F. Author et al.}
% First names are abbreviated in the running head.
% If there are more than two authors, 'et al.' is used.
%
\institute{LIACS, Leiden University, Leiden, Netherlands\\
\email{s3430863@umail.leidenuniv.nl}, \email{s3649024@umail.leidenuniv.nl}}
%
\maketitle              % typeset the header of the contribution
%
\begin{abstract}
%The abstract should briefly summarize the contents of the paper in 150--250 words.
To be continued.

\keywords{Sentiment analysis  \and BERT \and Ensemble learning}
\end{abstract}
%
%
%
\section{Introduction}

\section{Related Work}

\section{Data}

\section{Methodology}
\subsection{BERT}


    
\section{Experiments}
\subsection{Experimental Setup}

\subsection{Results}

\section{Discussion}
    
\section{Conclusion}
    
\section{Contributions}

% ---- Bibliography ----
%
% BibTeX users should specify bibliography style 'splncs04'.
% References will then be sorted and formatted in the correct style.
%
\bibliographystyle{splncs04}
\bibliography{paper}
%
\end{document}
